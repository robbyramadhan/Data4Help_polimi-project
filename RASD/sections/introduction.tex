\documentclass[../main.tex]{subfiles}

\begin{document}

\subsection{Purpose}

%TODO - needs revision
We are designing Data4Help, a software system capable of (sending and?) receiving informations about our customers, collected trough the use of wearable devices, as accurately as the currently widely available technology permits.
Such information includes current location, in terms of geographical position, and body parameters, such as heartbeat rate, blood pressure, etc.

We believe this data has great value, so we'll make it available for other companies to use for their purposes; they will only need to file a request to our system that, when approved, will guarantee access to our information, currently collected and yet to be, regarding a specific customer or to a group of people satisfying certain criteria.

Moreover, we will offer two other services that will make use of this data base.

The first one is AutomatedSOS, that will target mainly elderly people. Our software will constantly monitor their health status, and in case of emergency will promptly dispatch an ambulance to their current location, notifying the customer (and his/her relative/s?) of its arrival (and ETA?).

The second one is Track4Run, which will allow the organizers of a running event to define the path for it, the athletes to enroll in the event, and the spectators to track the position of the runners on a map in real time.


\subsubsection{Goals}

\begin{minipage}{\textwidth}
{\bf Data4Help}
\begin{description}
	\item [G1]  Third parties can make requests for accessing single customer data.
	\item [G2]  Third parties can make requests for accessing aggregate anonymous data specifiying filters.
	\item [G3]  Third parties can access stored data, for which a request has been approved.
	\item [G4]  Third parties can subscribe to new data, for which a request has been approved.
	\item [G5]  Users can accept or refuse requests for their data from third parties.
	\item [G6]  Third parties whose requests are accepted are not prevented from accessing data.
\end{description}
\end{minipage}
\vspace{8mm}

\begin{minipage}{\textwidth}
{\bf AutomatedSOS}
\begin{description}
	\item [G7]  From the time a customer's parameters indicate a health emergency status, an ambulance is dispatched to his location in less than 5 seconds.
	\item [G8]  When an ambulance is dispatched, the customer is notified of its arrival.
\end{description}
\end{minipage}
\vspace{8mm}

\begin{minipage}{\textwidth}
{\bf Track4Run}
\begin{description}
	\item [G9]   Organizers can create a new running event, prividing the required information.
	\item [G10]  Runners can view and enroll in available runs.
	\item [G11]  Spectators can see the position of participants on the map during a run.
\end{description}
\end{minipage}

\subsection{Scope}

The main purpose of Data4Help is creating a bridge between companies and individuals health status and location. As such, the main beneficiaries will be companies such as life insurances, hospitals and clinics, and universities for research purposes.
The focus of our two other services, AutomatedSOS and Track4Run, is instead on the single customer, mainly elderly people and runners respectively.

The software will collect customer's data through wearable devices; they'll need to either have GPS and an active internet connection per se, or a paired smartphone through which the customer will be localized and data will be sent and received via internet. We'll target Wear OS and Android as running platform. Companies on the other hand will be able to access our stored data through a REST web API.

% TrackMe is an application that will be used by two types of users:
%
% -Third party users who are interested to people’s data for some reason (e.g. Statistics).
%
% -Individual users that decide to give their information to the application and after positive answer to requests give it to third party users.
%
% The function “Data4Help” cover these scopes.
%
% With AutomatedSOS feature TrackMe also becomes useful for those people who want their vital parameters to be constantly monitored. In this way in case of emergency they can be assisted immediately thanks to an automatic call to the ambulance.
%
% The last feature is called “Track4Run”, it allows users to organize running competition or to enroll to them. The users can also localize all participant of a running competition.
% The data will be collected through a wearable device like a smartwatch and will send to personal smartphone where the application installed (Android or iOS) or the web app organize data and save it in a server to be used. The notification of requests will be sent to user’s smartphone.

\subsection{Definitions, acronyms, abbreviations}
\subsection{Revision history}
\subsection{Reference documents}

\begin{description}
	\item [The world \& the machine] - M. Jackson, P. Zave
\end{description}

\subsection{Document structure}

This document is structured in four main chapters, that are as follows:

\begin{enumerate}
	\item The first chapter serves as an introduction and an overview to the project, describing the main reasons for its development, what are its goals and giving a brief informat description of it.

	\item The second chapter serves as a more formal description of the project: it includes class diagrams, state machine diagrams, and it gives details on the shared phenomena and domain models. Class diagrams give a big picture description on how the system should be structured, while state machine diagrams focus on the more relevant entities of the model. Here are also presented all the requirements and domain assumptions the system in project must fulfill and take into considerations, in order to achieve the goals; they are presented each one after the goal it is relevant to.

	\item In the third chapter are presented the specific requirements, use cases described through the use of natural language and diagrams such as sequence/activity diagram, and the design constraints the system must satisfy. A mockup is also shown as a general idea of how the end product should be, in terms of design and functionalities offered to the end user.

	\item The fourth and last chapter is a formal analysis of the model, made through the use of the open source Alloy language and analyzer, including a graphic representation of it obtained from Alloy Tool.
\end{enumerate}

\end{document}
