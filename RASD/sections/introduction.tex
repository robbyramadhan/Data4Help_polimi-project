\documentclass[../main.tex]{subfiles}

\begin{document}

\vspace{8mm}
\subsection{Purpose}

%TODO - needs revision
We are designing Data4Help, a software system capable of (sending and?) receiving informations about our customers, collected trough the use of wearable devices, as accurately as the currently widely available technology permits.
Such information includes current location, in terms of geographical position, and body parameters, such as heartbeat rate, blood pressure, etc.

We believe this data has great value, so we'll make it available for other companies to use for their purposes; they will only need to file a request to our system that, when approved, will guarantee access to our information, currently collected and yet to be, regarding a specific customer or to a group of people satisfying certain criteria.

Moreover, we will offer two other services that will make use of this data base.

The first one is AutomatedSOS, that will target mainly elderly people. Our software will constantly monitor their health status, and in case of emergency will promptly dispatch an ambulance to their current location, notifying the customer (and his/her relative/s?) of its arrival (and ETA?).

The second one is Track4Run, which will allow the organizers of a running event to define the path for it, the athletes to enroll in the event, and the spectators to track the position of the runners on a map in real time.


\vspace{8mm}
\subsubsection{Goals}

\vspace{2mm}
{\bf Data4Help}
\begin{description}
	\item [G1]  Third parties can make requests for accessing single customer data.
	\item [G2]  Third parties can make requests for accessing aggregate anonymous data specifiying filters.
	\item [G3]  Third parties can subscribe to new data, for which a request has been approved.
	\item [G4]  Users can accept or refuse requests for their data from third parties.
	\item [G5]  Third parties whose requests are accepted are not prevented from accessing data.
\end{description}

\vspace{2mm}
{\bf AutomatedSOS}
\begin{description}
	\item [G6]  From the time a customer's parameters indicate a health emergency status, an ambulance is dispatched to his location in less than 5 seconds.
	\item [G7]  When an ambulance is dispatched, the customer is notified of its arrival.
\end{description}

\vspace{2mm}
{\bf Track4Run}
\begin{description}
	\item [G?]  Organizers can create a new event.
	\item [G8]  Organizers can define the path for a run.
	\item [G9]  Runners can view and enroll in available runs.
	\item [G10] Spectators can see the position of participants on the map during a run.
\end{description}

\vspace{8mm}
\subsection{Scope}

TrackMe is an application that will be used by two types of users:

-Third party users who are interested to people’s data for some reason (e.g. Statistics).

-Individual users that decide to give their information to the application and after positive answer to requests give it to third party users.

The function “Data4Help” cover these scopes.

With AutomatedSOS feature TrackMe also becomes useful for those people who want their vital parameters to be constantly monitored. In this way in case of emergency they can be assisted immediately thanks to an automatic call to the ambulance.

The last feature is called “Track4Run”, it allows users to organize running competition or to enroll to them. The users can also localize all participant of a running competition.
The data will be collected through a wearable device like a smartwatch and will send to personal smartphone where the application installed (Android or iOS) or the web app organize data and save it in a server to be used. The notification of requests will be sent to user’s smartphone.

\subsection{Definitions, acronyms, abbreviations}
\subsection{Revision history}
\subsection{Reference documents}
\subsection{Document structure}

\end{document}
