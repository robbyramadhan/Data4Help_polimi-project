\documentclass[../main.tex]{subfiles}

\begin{document}

\subsection{Product perspective}
\subsection{Product functions}
\subsection{User characteristics}
\subsection{Assumptions, dependencies, constrains}
{\bf Domain assumptions}

\begin{description}
    \item [G1]  Third parties can make requests for accessing single customer data.

        \begin{description}
            \item [D1]  Third parties are able to identify specific customers by SSN or FC.

        \end{description}
        
    \item [G2]  Third parties can make requests for aggregate anonymous data specifying filters.
        \begin{description}
            \item [D2]  A group is considered anonymous if is composed by more than 1000 people.

        \end{description}

    \item [G3]	Third parties can subscribe to new data, for which a request have been approved.
        \begin{description}
            \item

        \end{description}
        
    \item [G4]  Users can accept or refuse requests for their data from third parties.  
        \begin{description}
            \item [D3]  A sent request is always received by the targeted user

        \end{description}
    \item [G5]  Third parties who requests are accepted are not prevented to accessing data.
        \begin{description}
            \item

        \end{description}
    \item [G6]  From the time a customer's parameters indicate a health emergency status, an ambulance is dispatched to his location in less than 5 seconds.
        \begin{description}
            \item [D4] Health emergency status occurs when a AutomatedSOS customer parameters falls below or exceeds healthy boundaries.
            \item [D5] AutomatedSOS customers are always connected to GSM or internet.
            \item [D6] An ambulance is always available for dispatching and can reach the customer position.
            
        \end{description}
    \item [G7]  When an ambulance is dispatched, the customer is notified of its arrival.
        \begin{description}
            \item 

        \end{description}

    \item [G8]  Organizers can define the path for a run.
        \begin{description}
            \item [D7] Organizers choice a valid path and run are legally authorized by authority.

        \end{description}

    \item [G9]  Runners can view and enroll in created runs
        \begin{description}
            \item []

        \end{description}

    \item [G10] Spectators can see the position of participants on the map during a run.
        \begin{description}
            \item

        \end{description}
\end{description}

\end{document}