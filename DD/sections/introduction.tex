\documentclass[../main.tex]{subfiles}

\begin{document}

\subsection{Purpose}

This document provides a specification on the architecture of TrackMe's system. It is complementary to the Requirements Analysis and Specification Document already presented, and it provides further description on its components, their interactions, and the implementation, integration and testing plan.

\subsection{Scope}

The document will focus on illustrating the following components of the application and its development process:

\begin{itemize}

	\item{System architecture} 
	\item{UX} 
	\item{Implementation, integration and testing plan} 

\end{itemize}

\subsection{Definitions, Acronyms, Abbreviations}
\begin{description}
	\item{UX} User experience
	\item {RASD} Requirements Analysis and Specification Document
	\item {CIA triad} Confidentiality, Integrity, Availability; a model designed to guide policies for information security within an organization
	\item {API} Application Programming Interface
\end{description}

\subsection{Revision history}
\begin{tabular}{p{1.2cm}|p{2.1cm}|p{2.6cm}}
	\bf Version & \bf \makebox[2.1cm][c]{Release Date} & \bf \makebox[3cm][c]{Description} \\
	\hline
	\makebox[1.2cm][c]{1.0} & \makebox[2.1cm][c]{10/12/2018} & \makebox[3cm][c]{Initial Release}\\
\end{tabular}

\subsection{Reference Documents}

\begin{itemize}

	\item{Mandatory Project Assignment AY 2018/2019}

\end{itemize}

\subsection{Document Structure}

The structure of this document is the following:

\begin{enumerate}

	\item The first chapter serves as an introduction to this document.
	\item The second chapters details the architecture of TrackMe's system: its components, the way they interact with each other, and how are grouped together. It also comprises sequence diagrams that show in more detail the interactions of such components for some selected goal to be achieved. Futhermore, and the end of this chapter, can be found the architectural and design patterns that are used in the system, their purpose and the reason for choosing them, along with other design decisions.
	\item The third chapter futher describes the indications on the user interfaces and user experience provided in the RASD.
	\item In the fourth chapter is provided a map on which components of the system every requirement and goal specified in the RASD is achieved.
	\item The fifth chapter illustrates the plan for the implementation, integration and testing of the system to be developed.

\end{enumerate}

\end{document}
