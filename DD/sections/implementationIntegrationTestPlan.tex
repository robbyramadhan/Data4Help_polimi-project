\documentclass[../main.tex]{subfiles}

\begin{document}
\subsection{Implementation Plan}
TrackMe system is composed of five macro components (\textit{fig. 2}) that can be developed independently and simultanously
after interfaces and communications standards are properly choosen.\\



\subsection{Integration Plan}
Although macro components will be developed indipendently, is strongly suggested a top-down approach oriented to functionality
implementation
 \subsubsection{Mobile Client, Applications Servers and Database Server}
    The first things to be developed will be the "core functionality of the system" and Data4Help as shown in figure. Then it will be possible
    to test the backbone of the system (servers communication and APIs). Following the same approach, the development will focus consequently on
    AutomatedSOS and Track4Run.
 \subsubsection{Web Server}
    Being a totally independent part of the system, the Web Server will be developed by a different team or after the development of the other components.\\
    The APIs documentation to be hosted on the Web Server will be produced by the Data4Help Application server developers. 
\subsection{Test Plan}
    Following the top-down approach, each module should be tested as soon as it is completed.
\subsection{Programming Standard}
\subsection{Software Development Tools}
\subsubsection{Servers and Database}
Servers will be implemented using \textit{Node.js} a popular Javascript run-time environment with the \textit{Express.js} framework and \textit{Knex.js} as SQL query builder.\\
The database will use \textit{PostgresSQL} relational database system.
\subsubsection{Mobile Application}
Mobile Application will be developed on \textit{Flutter}, an open-source mobile application development SDK created by Google.

\end{document}
