\documentclass[../main.tex]{subfiles}

\begin{document}

\subsection{Implemented goals and requirements}

	All goals and requirements of Data4Help and AutomatedSOS have been implemented, at least partially, in the prototype.

	\vspace{1cm}

	{\bf Data4Help}

	\vspace{0.5cm}

	\begin{description}
		\item [G1]  Third parties shall be able to make requests for accessing single customer data.
		\item [G2]  Third parties shall be able to make requests for accessing aggregate anonymous data specifiying filters.
		\item [G3]  Third parties shall be able to access stored data, for which a request has been approved.
		\item [G4]  Third parties shall be able to subscribe to new data, for which a request has been approved.
		\item [G5]  Users shall be able to accept or refuse requests for their data from third parties.
		\item [G6]  Access to customers data from a third party, for which a request does not exist, or has been rejected, shall be denied.

		\item [R1] The system must forward any request from a company to single customer's data to the corresponding user.
		\item [R2] The system must reject any request for data regarding a group of customers when it cannot guarantee the anonimity of its components.
		\item [R3] The system must periodically collect and store customer's data.
		\item [R4] The system must periodically update the accessible data with the newly collected one
		\item [R5 *] The system must notify the user of the request and permit him to accept or refuse it.
		\item [R6] The system must update the request status with the answer provided by the user.
		\item [R7] The system must be able to identify which data can be accessed for each third party company.

	\end{description}

\vspace{8mm}


	{\bf AutomatedSOS}

	\vspace{0.5cm}

	\begin{description}
		\item [G7 *]  From the time a customer's parameters indicate a health emergency status, an ambulance shall be dispatched to his location in less than 5 seconds.
		\item [G8]  When an ambulance is dispatched, the customer shall be notified of its arrival.

		\item [R8] The system must continuously check the data read from customers subscribed to AutomatedSOS.
		\item [R9 *] In case the data indicate an emergency for a customer, the system must dispatch the closest ambulance to his location.
		\item [R10] After an ambulance is dispatched, the system must notify the customer of its arrival.

	\end{description}


\textbf{*} Partial implementation

\subsection{Partially or not implemented funcionalities}

Being this a prototype, not all funcionalities have been fully implemented. Following are the most notable ones:

\begin{itemize}
	\item All features of Track4Run are not implemented.
	\item The Mobile App component works and was tested only with the android Wear OS emulator. Being that on the emulator the sensors are not accessible, a simulator is provided with sliders to adjust the values sent to the mobile application.
	\item AutomatedSOS dispatching feature only works with the provided ambulance dispatcher simulator.
	\item The activation of the AutomatedSOS service is not implemented; instead, the service is assumed to be already activated on all the registered devices.
	\item Notification manager components are not implemented. This is not a problem for this protype though, as each resource that was to be notified is automatically synced through components accessing the proper funcionality (for example, new requests for data can be viewed by querying the system, and that's automatically done when opening the requests page on the Mobile App).
	\item There is no proper address or geolocation resolution and validation.
	\item Optimizations for large amount of data transfers, such as when sending user's data for a group request that includes many users, have not yet been implemented.
	\item Some minor funcionalities are left as todo in the source code, intended for completion on further development, and should not significantly impact the behaviour of the system, if at all.
\end{itemize}

\end{document}
